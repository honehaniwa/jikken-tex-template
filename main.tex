\documentclass[12pt, dvipdfmx,autodetect-engine]{jsarticle}
\usepackage[dvipdfm]{graphicx}
\usepackage{listings, settings/jlistings}
\usepackage{pdfpages}
\usepackage{color}
\usepackage{here}
\usepackage{bm}			% ベクトル太字(\bm{A}) or \mathbf
\usepackage{amsmath,amssymb}
\usepackage{mathtools}
\mathtoolsset{showonlyrefs=true}

\lstset{
  basicstyle={\ttfamily},
  identifierstyle={\small},
  commentstyle={\smallitshape},
  keywordstyle={\small\bfseries},
  ndkeywordstyle={\small},
  stringstyle={\small\ttfamily},
  frame={tb},
  breaklines=true,
  columns=[l]{fullflexible},
  numbers=left,
  xrightmargin=0zw,
  xleftmargin=3zw,
  numberstyle={\scriptsize},
  stepnumber=1,
  numbersep=1zw,
  lineskip=-0.5ex
}

\renewcommand{\lstlistingname}{ソースコード}
\makeatletter
\newcommand{\subsubsubsection}{\@startsection{paragraph}{4}{\z@}%
  {1.0\Cvs \@plus.5\Cdp \@minus.2\Cdp}%
  {.1\Cvs \@plus.3\Cdp}%
  {\reset@font\sffamily\normalsize}
}
\makeatother
\setcounter{secnumdepth}{4}

\begin{document}
% % 12ppで固定
% \large
% 表紙をfrontpage.pdfにしてプロジェクトに入れておけばOK
\includepdf[]{frontpage.pdf}

\section{目的}
\section{実行環境}
\section{原理}
\subsection{サブセクション}
\section{実験内容}
\section{実験結果}
\section{考察}
\newpage
\section{テンプレート集}
%mathjax使えるの偉い
$D$個の特徴量を用いた$D$次元のベクトル$x$を以下のように表す。
% 数式
\begin{equation}
	\mathbf{x} = (x_1, x_2, ..., x_D)^T \notag
\end{equation}
\begin{itemize}
    \item 実行OS:Windows 10
    \item データ変換用言語: Python 3.8.5
    \item データ変換用ライブラリ: pandas 1.1.3
\end{itemize}
図\ref{fig:model}で図番号自動指定できる
\begin{figure}[h]
	\centering
	\includegraphics[width=8cm]{fig/takane.png}
	\caption{(図題)貴音さんはいいぞ}
	\label{fig:model}
\end{figure}
% ソースコード
\lstinputlisting [caption=ソースコード, label=adder]{code.py}
\begin{table}[H]
	\caption{表題}
	\label{table:car_data}
	\centering
	\begin{tabular}{|l|c|r|}
		\hline
		左寄せ&		中央寄せ&		右寄せ\\ \hline\hline
		aaaa  &		bbbbb &     ccccc	\\ \hline
		AtCoder  &		 Codeforces     &	topcoder 		\\ \hline
	\end{tabular}
\end{table}
\newpage
% 参考文献
\begin{thebibliography}{7}
    \bibitem{sankou01} Erich Gamma, Richard Helm, Ralph Johnson, and John Vlisside, "Design Patterns: Elements of Reusable Object-Oriented Software," Addison-Wesley, 1994.
	\bibitem{sankou02} Geoffrey Hinton and Sam Roweis, "Stochastic Neighbor Embedding," Advances in Neural Information Processing Systems, Vol. 15, pp. 833840, 2002.
	\bibitem{sankou03} Laurans van der Maaten and Geoffrey Hinton, "Visualizing Data using t-SNE," Journal of Machine Learning Research, Vol. 9, pp. 25792605, 2008.
	\bibitem{sankou04} John W. Sammon, JR., "A Nonlinear Mapping for Data Structure Analysis," IEEE Transactions on Computers, Vol. C-18, No. 5, pp. 401409, 1969.
	\bibitem{sankou05} Joshua B. Tanenbaum, Vin de Silva, and John C. Langford, "A Global Geometric Framework for Nonlinear Dimensionality Reduction," Science, Vol. 290, pp. 23192323, 2000.
	\bibitem{sankou06} Warren S. Torgerson, "Multidimensional Scaling: I. Theory and Method," Psychometrika, Vol. 17, pp. 401419, 1952.
	\bibitem{sankou07} Jarkko Venna and Samuel Kaski, "Local multidimensional scaling," Neural Networks, Vol.19, pp. 889899, 2006.
	\bibitem{sankou08}  『Java言語で学ぶ デザインパターン入門』(結城浩 ソフトバンクパブリッシング株式会社出版 2001年)
	\bibitem{sankou09} Car evaluation Data Set, UCI Machine Learning Repository (URL: https://archive.ics.uci.edu/ml/datasets/Car+Evaluation) 2020年1月30日
\end{thebibliography}


\end{document}
